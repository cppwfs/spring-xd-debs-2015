\section{Analytics}
\label{sec:Analytics}

Spring XD supports analytics functionality through various sink and processor modules.
Typically a ``main'' stream which obtains data from an external source (via
a source module) and writes it to an external data store (via a sink module)
is ``tapped'' -- meaning that a separate stream is created using the data
flowing from the first stream. The existence of this tap is unknown by the
main stream. This allows for analytics to be performed in a non-invasive manner.
The results of the analytics can be written to another sink module or a
subsequent processing module for further analysis.

\par

Spring XD includes analytics modules such as counter, field value counter,
aggregate counter, gauge and rich gauge. It also provides support for running
predictive analytics using PMML\cite{pmml} model scoring. Spring XD also includes a module
for executing arbitrary shell commands. This shell module can execute an R\cite{r-language}
or Python\cite{python-language} based processor or sink implementation which could perform
dynamic model scoring. Additionally, a Spark MLlib job may be executed as a Spring XD job,
benefiting from Spring XD's capability to monitor and manage the job. See section~\ref{sec:Spark}
for more information on Spark integration.

\subsection {Basic Analytics}

To perform some of the basic analytic operations, Spring XD provides the following
modules. Since these modules write the analytic results to a data store, they are
considered sinks. Currently in-memory and Redis\cite{redis} are supported data stores. This
data is exposed by the admin server via the REST API. This allows for simple
development of an application to gain access to the analytic data results.

\subsubsection {Counter}

This module counts the number of events triggered on various stages of a stream.
This can be used to tap a main stream at various stages and count the number
of messages sent out at each stage.

\subsubsection {Field Value Counter}

This module counts the number of occurrences of a specific field from a message
flowing through a stream. Spring XD supports the following message payload types
out of the box: POJO (Java Bean), tuple and JSON\cite{json} string.

\subsubsection {Aggregate Counter}

The aggregate counter is similar to the simple counter but also keeps track of
the time period. Total count values for each minute, hour, day and month
of the period in which data was collected may be queried.

\subsubsection {Gauge}
Gauge is a metric that represents a single long value associated with a unique name.
In Spring XD, the gauge metric sink expects a numeric value as a payload.

\subsubsection {Rich Gauge}
Rich Gauge is a metric that holds a double value associated with a unique name. In
addition to the value, this metric keeps a running average along with the minimum and
maximum values and the count.

\subsection {Predictive Analytics using PMML}
Spring XD provides support for running real time model scoring using JPMML-Evaluator.
This evaluator supports a wide range of model types and is interoperable with
models exported from R, Rattle, KNIME, and RapidMiner. Once the analytical model
is defined as a PMML model, the evaluator can run model scoring based on the 
input data being ingested.

\subsection {Shell Command Support}
Spring XD includes processor and sink modules for invoking external shell commands.
A processor or sink module can be implemented to run predictive
analytics algorithms written in other languages such as Python, R or any other language that
supports shell command invocation. This support allows for dynamic updating of
an analytical model based on input data, therefore making predictive analytics more adaptive.

\subsection{Spark MLlib}
A Spark MLlib algorithm can be run as a `sparkApp' job in Spring XD. This makes the
orchestration of Spark job workflow, monitoring and management easier.
A Spark MLlib job can also be triggered by a stream with data that
would act as the input for the MLlib algorithm. This closed loop
integration (integration streaming data to trigger the job) is
beneficial for predictive analytics.
