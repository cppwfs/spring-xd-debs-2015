\section{Introduction}

The era of Big Data has introduced many new technologies for data storage
and processing. The demand for these solutions is being driven by high
volumes of data that need to be captured and analyzed. These data sources
include (but are not limited to) mobile devices, vehicles (both autonomous
and manually operated) and large scale applications.

Most data processing applications fall into one of two categories:
streaming or batching. Streaming applications are used to capture data
in real time, typically as the data is being generated. Batching applications
operate on data that has already been captured to persistent storage. Batching
works best when throughput is more important than latency.

These requirements have led to the creation of new projects such as Apache Hadoop~\cite{hadoop},
Apache Spark~\cite{spark}, and Apache Kafka~\cite{kafka}. Each of these projects was conceived
to solve a specific data processing and/or storage requirement. While some
of these projects overlap, they are often complementary - for example,
a streaming application that collects event data in Kafka in real-time and also 
stores it into Hadoop Distributed File System (HDFS)~\cite{hdfs} for batch processing.

Integrating the best of breed solutions is a challenge for developers
interested in building streaming and/or batching applications. Writing
code to glue together a custom solution imposes an upfront cost,
as well as the cost of ongoing maintenance.

Spring XD was created to help developers and system operators overcome
these challenges. Spring XD provides the following features that improve developer
productivity and operational efficiency for streaming and/or batching
data applications:
\begin{itemize*}
\item A distributed runtime environment to manage streaming and batching applications
\item Out of the box integration with a plethora of technologies, both established
and up-and-coming
\item An interactive shell for creating streams or jobs without writing any Java code
\end{itemize*}

Although enterprises are interested in adopting these new technologies to
meet their data demands, they have already invested heavily in older and proven
technologies such as SQL databases and messaging systems. Spring XD includes
integrations with these technologies, thus simplifying the connection between
existing systems with newer ones.

Creating a stream in Spring XD is a simple concept for those familiar with
UNIX streams and pipes. Consider the following shell command:

\verb;tail -f /tmp/log.txt | grep ERROR;

The \texttt{tail} command will continuously display the file contents. The |
will pipe the output of \texttt{tail} to \texttt{grep}, which will filter 
out all lines that do not contain the string ERROR.

The equivalent using Spring XD looks like this:

\verb;stream create -name error-filter -definition;\\*
\verb;  "tail -name=/tmp/log.txt | filter;\\*
\verb;  --expression=payload.contains('ERROR') | log";

While this specific example will only tail a local file, a distributed 
ingestion stream that aggregates, filters, and stores log file analytics
can just as easily be created with Spring XD.
