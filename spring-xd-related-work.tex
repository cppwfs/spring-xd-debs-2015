\section{Related Work}
This section compares and contrasts Spring XD to similar projects.

\subsection{Spring XD and Spark}
Spark is a general-purpose framework for large scale data processing.

For customers who need a rock solid ingestion platform based on mature open source technology, analytical stream APIs for the 80 percentage use case, RESTful control over stream processing and batch jobs across a range of big data technologies, as well as simplifying development and testing of Spark applications, Spring XD is available and supported today.

The following features differentiate Spring XD and Spark Streaming.

\begin{itemize*}
\item Reliably operate 24/7.
\item Ingest data into HDFS following to best practices (no small files) and with zero code.
\item Microbatching based on event count.
\item One event at a time processing.
\item Specify specific hosts where calculations are performed.
\item Decouple infrastructure code from business logic\slash analytics.
\end{itemize*}

The following features differentiate Spring XD and Spark Batch processing.

\begin{itemize*}
\item Provider a REST API and lifecycle management of Spark jobs.
\item Integrate with other Batch systems.
\end{itemize*}

In the future Spring XD will continue to provide additional integration with Spark features such as MLLib and SparkSQL.

\subsection{Spring XD and Storm}
Storm is a distributed computation system for real time stream processing.

For customers who are in need of real time stream processing, distributed data computations, and ETL capabilities, Spring XD as an unified runtime supports variety of use cases ranging from classic enterprise to Big Data and IoT. 

The following features differentiate Spring XD and Storm.

\begin{itemize*}
\item Spring XD's Shell compared with Storm's API programming paradigm delivers immediate productivity. No coding, IDE, bundling or packaging necessary. The high level DSL abstracts complexities through developer friendly fixtures.
\item REST APIs allows topology to be built in isolation without having to disrupt existing pipelines.
\item Loosely coupled `modules' that are responsible for ingestion, analytics, data processing, machine learning or data export can be individually managed and dynamically scaled.
\item Simplified governance model for `modules' (unit\-of\-work) and colocation capabilities.
\item Composition of `modules' (unit\-of\-work) improves performance characteristics. 
\item Building upon the reactive streams specification, customers have the option to choose from Spring Reactor, Spark Streaming or RxJava functional APIs, to build complex data centric applications.
\end{itemize*}

Spring XD provides you the right tool for the job, not forcing unnecessary complexity while remaining open and extensible.

\subsection{Spring XD and Flume}
Flume is a distributed system for collecting, aggregating and moving large data sets. 

For customers who need collecting and moving data, Spring XD simplifies lifecycle management, coordination, and automation of data pipelines. A high level DSL is all you need to operationalize data movements. 

The following features differentiate Spring XD and Flume.

\begin{itemize*}
\item Nothing manual and no coding! The high level DSL abstracts configuration and setup. One line DSL declaration automates data collection, processing, and export to a desired format or destination.
\item Manage data centric workflows with REST APIs either via DSL, admin ui, or custom applications.
\item Administer or monitor data pipelines through the UI, Jolokia or JMX endpoints. 
\item >500 combinations of data pipelines/flows can be created OOTB.
\item Granular control options to intercept step execution of data pipeline to create complex data driven workflows.
\item Flexibility through `Deployment Manifest' to declaratively configure data partitioning strategy to route data to a specific consumer instance in the cluster. Improves performance characteristics by virtue of colocation.
\end{itemize*}

No more config files, IDE, bundling, packaging or deploying artifacts. Spring XD's developer friendly fixtures deliver productivity. Flume offers HBase, Solr, and ElasticSearch sinks along with encryption support for Avro sources, which we are planning to address in our future releases.

\subsection{Spring XD and Oozie}
Oozie is a workflow scheduler engine to manage Hadoop workloads such as MapReduce or Pig jobs. 

For customers who need workflow or a coordinator engine, Spring XD, as a unified platform provides orchestration of directed graph processes to govern not just Hadoop but any data workflows. 

The following features differentiate Spring XD and Oozie.

\begin{itemize*}
\item Building upon Spring Batch, a JSR standardization (JSR-352) of batch workload data processing, Spring XD inherits rich enterprise features and customer trust.
\item Ready to use frequently used workflow jobs such as file-to-jdbc, file-to-hdfs, ftp-to-hdfs, hdfs-to-jdbc, hdfs-to-mongo, jdbc-to-hdfs, spark-job, and sqoop-job.
\item Flexibility to create custom workflow jobs through REST APIs.
\item Dynamic scaling of OOTB workflow-jobs and custom-workflow-jobs without having to bring down the runtime cluster or the currently running topology.
\item Runtime offers bidirectionality between real-time streaming and offline batch workflows to accommodate complex data processing use cases.
\item Ability to create and launch workflow-jobs from admin ui. Historical snapshots of execution, errors and states are available for exploration via admin ui.
\item Portable runtime that can run where there is JVM. On-prem, Pivotal CloudFoundry, YARN, Mesos, Docker or EC2.
\end{itemize*}

Bridging the gaps between offline and near real time data, Spring XD will continue to evolve as one stop platform for data workflows to efficiently process the data in-transit, at-rest, and in-use. Oozie offers HCatalog integration, which we are planning to address in our future releases.

\subsection{Spring XD and Sqoop}
Sqoop assists with data transmission between Hadoop and relational databases.

For customers who need data ingest and export between various databases and Hadoop, Spring XD's unified platform facilitates out of the box fixtures to orchestrate data movements reliably. 

The following features differentiate Spring XD and Sqoop.

\begin{itemize*}
\item Comprehensive lifecycle management platform for not just data transmission between Hadoop and Databases but also real time data ingest, analytics and export use cases.
\item Extending the workflow infrastructure to write custom tasklets is straightforward.
\item Simplified linux style semantics to create, deploy and destroy data pipelines.
\item Flexibility to introduce custom data pipelines through REST APIs.
\item Unified functional programming runtime to support reactive streams specifications. Either Spring Reactor, Spark Streaming or RxJava functional APIs can be used to build reactive data pipelines.
\end{itemize*}

Sqoop offers data validation, data merge, incremental data imports, and HCatalog integration among others. As a unified platform, Spring XD provides an OOTB Sqoop job to take advantage of the features at the same time also efficiently orchestrate lifecycle of data workload use cases efficiently. 
