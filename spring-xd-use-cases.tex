\section{Use Cases}
\label{sec:Use Cases}

The following customer use cases are being supported in the field by the
Spring XD team.

\subsection{Fault Detection}
\textit{Challenge}: A large equipment manufacturer is interested in
ingesting machine data to apply predictive analytics to proactively monitor
performance and adjust business operations.

\textit{Solution}: Given the extensible architecture in Spring XD, a custom
source module was created to handle proprietary data formats, thus allowing
consumption and transformation of data complying with their proprietary
standards. The predictive models, complying with PMML specification, were
generated based on historical trends. Spring XD's analytic-pmml processor was
used in a stream to intercept machine events to compute predictions in
real-time. The outliers were captured in Redis\cite{redis} for dashboard alerts 
and ad-hoc data analysis via REST.

\subsection{Enterprise Modernization}
\textit{Challenge}: A large retail provider is heavily invested in the Hadoop
\cite{hadoop}
ecosystem. They weren't keen on maintaining or supporting several products in
production - operationalizing this was painful. They wanted to streamline and
modernize their Hadoop workflows.

\textit{Solution}: As one-stop runtime, Spring XD provides dozens of data
integration adapters to send and receive data from external applications, thus
allowing the customer to use the unified approach for all ingestion use-cases.
Building upon Spring Batch, Spring XD also provides multiple batch jobs for data
movements along with granular workflow steps. This further eliminated the need
for relying on different workflow and data movement tools. Whether it is 
MapReduce, Hive, Pig, or HBase scripts -- the developer experience is the same.

\subsection{Data Ingest}
\textit{Challenge}: A startup in San Francisco was looking for a solution to
unify stream and batch operations. 

\textit{Solution}: Spring XD is used as a standard tool for ingesting data 
into Hadoop . Whether it is real-time (i.e., online) or batch 
(i.e., offline),
the out of the box fixtures delivered immediate benefits. Overall, the data
integration, orchestration, and data movement capabilities were handled end
to end by Spring XD.

\subsection{24/7 Production Pipelines}
\textit{Challenge}: A large IT enterprise wanted to build data pipelines to
monitor current software and hardware sales in order to predict and forecast.
Building pipelines in one; running them reliably 24/7 with strict SLAs is
another big challenege.

\textit{Solution}: Spring XD is built as highly-available runtime. Automatic
recovery from fault scenarios along with reprocessing of data events fulfilled
SLAs guarantees. Production running pipelines in Spring XD are long running
processes. For ad-hoc operations such as querying, machine learning, or
data crunching, `taps' in Spring XD were adopted to fork the data from primary
pipeline. This results in no disruption with the primary pipeline, at the
same time satisfying ad-hoc demands.

\subsection{Closed-loop Analytics}
\textit{Challenge}: A finance institution wanted to build a platform to detect
fraudulent transactions. They had years of historical data in varying formats
and the transactions happening in real-time at high volume. They wanted to
automate generation of historical models from raw dataset and apply the models
to real-time events. No in-house talent to build one; neither had budget to buy
expensive products.

\textit{Solution}: A batch job was used to learn from historical data thus
producing predictive models complying with PMML specification. The last step
in the batch workflow is orchestrated to deliver the generated models to
streams that ingest for real-time predictions. The predictions were persisted
in in-memory data grids to feed the dashboard. Spring XD, thus eliminating
top-talent hires to operationalize analytics pipeline.
